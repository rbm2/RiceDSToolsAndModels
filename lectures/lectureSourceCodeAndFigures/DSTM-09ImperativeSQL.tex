%Copyright 2019 Christopher M. Jermaine (cmj4@rice.edu) and Risa B. Myers (rbm2@rice.edu)
%
%Licensed under the Apache License, Version 2.0 (the "License");
%you may not use this file except in compliance with the License.
%You may obtain a copy of the License at
%
%    https://www.apache.org/licenses/LICENSE-2.0
%
%Unless required by applicable law or agreed to in writing, software
%distributed under the License is distributed on an "AS IS" BASIS,
%WITHOUT WARRANTIES OR CONDITIONS OF ANY KIND, either express or implied.
%See the License for the specific language governing permissions and
%limitations under the License.
\documentclass[aspectratio=169]{beamer}
\mode<presentation> 
{
\usetheme[noshadow, minimal,numbers,riceb,nonav]{Rice}
\usefonttheme[onlymath]{serif}
\setbeamercovered{transparent}
}
\useinnertheme{rectangles}
\usepackage{colortbl}

\usepackage[english]{babel}

\usepackage{mathptmx}
\usepackage{helvet}
\usepackage{courier}
\usepackage[T1]{fontenc}
\usepackage{trajan}
\usepackage{ textcomp }
\usepackage{amssymb}
\usepackage{tikz}   


\usepackage{mathptmx}
\usepackage{helvet}
\usepackage{courier}

\usepackage[T1]{fontenc}
\usepackage{trajan}
\usepackage{ textcomp }
\usepackage{listings}

\newenvironment{noindentitemize}
{ \begin{itemize}
  \setlength{\itemsep}{1.5ex}
  \setlength{\parskip}{0pt}
  \setlength{\parsep}{0pt}   
  \addtolength{\itemindent}{-2em}  }
{ \end{itemize} }

\newenvironment{noindentitemize2}
{ \begin{itemize}
  \setlength{\itemsep}{0ex}
  \setlength{\parskip}{0pt}
  \setlength{\parsep}{0pt}   
  \addtolength{\itemindent}{-2em}  }
{ \end{itemize} }


\lstnewenvironment{SQL}
  {\lstset{
        aboveskip=5pt,
        belowskip=5pt,
        escapechar=!,
        mathescape=true,
        upquote=true,
        language=SQL,
        basicstyle=\linespread{0.94}\ttfamily\footnotesize,
        morekeywords={PRINT, CURSOR, OPEN, FETCH, CLOSE, DECLARE, BEGIN, END, PROCEDURE, FOR, EACH, WITH, PARTITION, TEST, WHETHER, PROBABILITY, OUT,LOOP,IF,CONTINUE, HANDLER,CALL, FUNCTION, RETURNS, LANGUAGE,BODY,RETURN, REPLACE,plpgsql,
        EXIT, TEXT, REFCURSOR, QUOTE_LITERAL, DELIMITER,CONCAT,FOUND,LEAVE },
        deletekeywords={VALUE, PRIOR},
        showstringspaces=true}
        \vspace{0pt}%
        \noindent\minipage{0.47\textwidth}}
  {\endminipage\vspace{0pt}}

\lstnewenvironment{SQLtiny}
  {\lstset{
        aboveskip=5pt,
        belowskip=5pt,
        escapechar=!,
        mathescape=true,
        upquote=true,
        language=SQL,
        basicstyle=\linespread{0.94}\ttfamily\tiny,
        morekeywords={PRINT, CURSOR, OPEN, FETCH, CLOSE, DECLARE, BEGIN, END, PROCEDURE, FOR, EACH, WITH, PARTITION, TEST, WHETHER, PROBABILITY, OUT,LOOP,IF,CONTINUE, HANDLER,CALL, FUNCTION, RETURNS, LANGUAGE,BODY,RETURN,REPLACE,plpgsql,
        EXIT, TEXT, REFCURSOR, QUOTE_LITERAL, DELIMITER,CONCAT,FOUND,LEAVE },
        deletekeywords={VALUE, PRIOR},
        showstringspaces=true}
        \vspace{0pt}%
        \noindent\minipage{0.47\textwidth}}
  {\endminipage\vspace{0pt}}

  
\newcommand{\LIKES}{\textrm{LIKES}} 
\newcommand{\FREQUENTS}{\textrm{FREQUENTS}} 
\newcommand{\SERVES}{\textrm{SERVES}} 
\newcommand{\CAFE}{\textrm{CAFE}} 
\newcommand{\COFFEE}{\textrm{COFFEE}} 
\newcommand{\DRINKER}{\textrm{DRINKER}} 
\newcommand{\ALLPEEPS}{\textrm{ALLPEEPS}} 
\newcommand{\ALLCOMBOS}{\textrm{ALLCOMBOS}} 
\newcommand{\NOGOODCOFFEE}{\textrm{NOGOODCOFFEE}} 


\setbeamerfont{block body}{size=\tiny}


%===============================================================%

\title[]
{Tools \& Models for Data Science}

\subtitle{Imperative SQL}

\author[]{Chris Jermaine \& Risa Myers}
\institute
{
  Rice University 
}

\date[]{}


\subject{Beamer}


\begin{document}


\begin{frame}
 \titlepage
\end{frame}



%***********************************************************
\begin{frame}{In SQL, We Can Write Imperative Code}

\begin{itemize}
\item[?] Why is this useful?
\end{itemize}
\end{frame}
%***********************************************************
\begin{frame}{In SQL, We Can Write Imperative Code}

\begin{itemize}
\item Why useful?
	\begin{itemize}
		\item Encapsulation --- make it easy for the programmer
		\item Safety --- protect the database from the programmer
		\item Performance --- fewer end-to-end trips
		\item Reuse
		\item Can process records sequentially
		\item Can respond to events
	\end{itemize}
\end{itemize}
\end{frame}

%***********************************************************
\begin{frame}{Stored Procedure / Function}

\begin{itemize}
\item Common form of imperative code: stored procedure and / or function 
	\begin{itemize}
	\item Code stored in the DB
	\item Can be invoked from
	\begin{itemize}
	\item A query
	\item An external program
	\item Another stored procedure / function
	\item A trigger
	\end{itemize}
	\end{itemize}
\end{itemize}
\end{frame}

%***********************************************************
\begin{frame}[fragile]{Basic Form}

\begin{SQL}
CREATE FUNCTION myFunc(/* list params */)
   RETURNS <datatype> AS
$\textbf{\$\$}$
DECLARE
/* declarations here */
BEGIN
/* code here */
END;
$\textbf{\$\$}$
LANGUAGE plpgsql ;
\end{SQL}

\end{frame}

%***********************************************************

\begin{frame}[fragile]{First Stored Procedure Example}

\begin{SQL}
PEAK (NAME, ELEV, DIFF, MAP, REGION)
\end{SQL}

\begin{itemize}
\item[?] Write a stored procedure to get the peak count in a region
	\begin{itemize}
	\item But if no region given...
	\item Return the count overall
	\end{itemize}
\end{itemize}
\end{frame}
 
%***********************************************************

\begin{frame}[fragile]{First Stored Procedure Example}

\begin{SQL}
PEAK (NAME, ELEV, DIFF, MAP, REGION)

CREATE OR REPLACE FUNCTION getNumPeaks(whichRegion CHARACTER VARYING)
  RETURNS INTEGER AS
$\textbf{\$\$}$
DECLARE
	/* declarations here */
BEGIN
	/* code here */
END;
$\textbf{\$\$}$
LANGUAGE plpgsql;\end{SQL}
\end{frame}

%***********************************************************

\begin{frame}[fragile]{First Stored Procedure Example}

\begin{SQL}
CREATE OR REPLACE FUNCTION 
   getNumPeaks(whichRegion character varying)
   RETURNS integer AS
$\textbf{\$\$}$
DECLARE
  queryString TEXT;
  numPeaks INTEGER;
BEGIN
  numPeaks = 0;
  -- build the query
  queryString = 'SELECT COUNT(*) FROM peak ' $||$ 
      COALESCE (' WHERE Region=' $||$ 
         QUOTE_LITERAL(whichRegion) ,'');
  -- run the query
  EXECUTE queryString INTO numPeaks;
  RETURN numPeaks;
END;
$\textbf{\$\$}$
LANGUAGE plpgsql;
\end{SQL}
\end{frame}
%***********************************************************

\begin{frame}[fragile]{Thoughts on First Stored Procedure Example}


\begin{itemize}
	\item All local variables need to be declared
	\item Code is bounded by $\textbf{\$\$}$ 
	\item Language is specified at the end
	\item[?] What's the deal with \texttt{QUOTE\_LITERAL}?
	\item[?] What's \texttt{COALESCE}?
	\item[?] \texttt{EXECUTE}: common, powerful, dangerous! Why?
\end{itemize}
\end{frame}

%***********************************************************
\begin{frame}[fragile]{First Stored Procedure Example}

\begin{noindentitemize}
	\item[] Then to call
\end{noindentitemize}

\begin{SQL}
SELECT * FROM getNumPeaks('Corocoran to Whitney');
\end{SQL}

and 

\begin{SQL}
SELECT * FROM getNumPeaks(NULL);
\end{SQL}
\end{frame}


%***********************************************************
\begin{frame}[fragile]{Next Stored Procedure Example}

\begin{itemize}
\item Like before, but now we'll use a cursor
\end{itemize}

\end{frame}

%***********************************************************
\begin{frame}[fragile]{Next Stored Procedure Example}

\begin{SQL}
CREATE OR REPLACE FUNCTION 
   getNumPeaks(whichRegion CHARACTER VARYING) RETURNS INTEGER AS
$\textbf{\$\$}$
DECLARE
  myCursor REFCURSOR;
  queryString TEXT;
  numPeaks INTEGER;
BEGIN
  numPeaks = 0;
  -- build the query
  queryString = 'SELECT COUNT(*) FROM peak ' || COALESCE (' WHERE Region=' || 
       QUOTE_LITERAL(whichRegion) ,'');
  -- run the query
  OPEN myCursor FOR EXECUTE queryString;
  FETCH  myCursor INTO numPeaks;
  CLOSE myCursor;
  RETURN numPeaks;
END;
$\textbf{\$\$}$
LANGUAGE plpgsql;
\end{SQL}
\end{frame}

%***********************************************************
\begin{frame}[fragile]{Next Stored Procedure Example}

\begin{SQL}
BEGIN
  numPeaks = 0;
  -- build the query
  queryString = 'SELECT COUNT(*) FROM peak ' || COALESCE (' WHERE Region=' || 
       QUOTE_LITERAL(whichRegion) ,'');
  -- run the query
  OPEN myCursor FOR EXECUTE queryString;
  FETCH  myCursor INTO numPeaks;
  CLOSE myCursor;
  RETURN numPeaks;
END;
\end{SQL}

\begin{itemize}
\item What's new here: a ``cursor''
\begin{itemize}
	\item Standard abstraction for dealing with record sets
	\item Essentially an iterator
\end{itemize}
\end{itemize}
\end{frame}
%***********************************************************
\begin{frame}[fragile]{Steps for Using a Cursor}
\begin{enumerate}
\item Declare
\item Open
\item Loop
\begin{enumerate}
\item[3.1] Fetch 
\item[3.2] Check for done
\item[3.3] Repeat from 3.1
\end{enumerate}
\item Close
\end{enumerate}
\end{frame}

%***********************************************************
\begin{frame}[fragile]{Cursor Options / Variations}
\begin{itemize}
\item \texttt{FETCH / MOVE} options
\begin{itemize}
\item \texttt{NEXT / PRIOR}
\item \texttt{FIRST / LAST}
\item \texttt{ABSOLUTE / RELATIVE} n
\item \texttt{FORWARD / BACKWARD} n
\item ...
\end{itemize}
\item Explicit vs. Implicit
\item[] \texttt{FOR} rowData \texttt{IN SELECT ... LOOP ... END LOOP}
\item[] \texttt{SELECT INTO}...
\end{itemize}
\end{frame}

%***********************************************************
\begin{frame}[fragile]{How to Debug Imperative SQL}
\begin{itemize}
\item It's primitive
\item Use \texttt{RAISE NOTICE} \textquotesingle{Step 1}\textquotesingle;
\item Use \texttt{RAISE NOTICE} \textquotesingle{\%}\textquotesingle, <var>;
\item e.g. \texttt{RAISE NOTICE} \textquotesingle{\%}\textquotesingle, queryString;
\item Start simple 
\item Consider writing data to an external (temporary table)
\end{itemize}
\end{frame}


%***********************************************************

\begin{frame}{Wrap up}
\begin{itemize}
	\item[?] How can we use what we learned today?
	\vspace{2em}
	\item[?] What do we know now that we didn't know before?
\end{itemize}

\end{frame}

\end{document}
