%Copyright 2019 Christopher M. Jermaine (cmj4@rice.edu) and Risa B. Myers (rbm2@rice.edu)
%
%Licensed under the Apache License, Version 2.0 (the "License");
%you may not use this file except in compliance with the License.
%You may obtain a copy of the License at
%
%    https://www.apache.org/licenses/LICENSE-2.0
%
%Unless required by applicable law or agreed to in writing, software
%distributed under the License is distributed on an "AS IS" BASIS,
%WITHOUT WARRANTIES OR CONDITIONS OF ANY KIND, either express or implied.
%See the License for the specific language governing permissions and
%limitations under the License.
\documentclass{article}

% General document formatting
\usepackage[margin=0.7in]{geometry}
\usepackage[parfill]{parskip}
\usepackage[utf8]{inputenc}
\usepackage{multirow}
\usepackage{xcolor,colortbl}

% for copyright notice
\usepackage{textcomp}

% Related to math
\usepackage{amsmath,amssymb,amsfonts,amsthm}

\usepackage{fancyhdr}
\pagestyle{fancy}

% clear the default header and footer
\fancyhf{}
\lfoot{\textcopyright  2019 Christopher M. Jermaine (cmj4@rice.edu) and Risa B. Myers (rbm2@rice.edu)}
\rfoot{\thepage}


\title{Relational Calculus Handout}
\date{}

% Set this length to 0pt if you don't want any lines!
\renewcommand{\headrulewidth}{0pt}

% Apply the fancy header style
\begin{document}
\maketitle
\thispagestyle{fancy}

FREQUENTS (DRINKER, CAFE)

SERVES (CAFE, COFFEE)


FREQUENTS\\
\begin{tabular}{|l|c|c| }  \hline
\textrm{DRINKER} & \textrm{CAFE}\\ \hline
Chris & A \\ \hline
Chris & B\\ \hline
Chris & C  \\ \hline
Risa & A \\ \hline
Risa & B \\ \hline
\end{tabular}


SERVES\\
\begin{tabular}{|l|c|c| }  \hline
\textrm{CAFE} & \textrm{COFFEE}\\ \hline
A & Drip\\ \hline
A & Cold Brew \\ \hline
A &  Espresso \\ \hline
B & Drip \\ \hline
C & Espresso \\ \hline
\end{tabular}


We want to know:

\textbf{Who has not gone to a cafe serving Cold Brew?}

Earlier, we established that we can answer the question:

\textbf{Who has gone to a cafe that serves Cold Brew?}

with 

$\{f.\textrm{DRINKER } | \textrm{ FREQUENTS}(f) \wedge \exists(s)(\textrm{SERVES}(s)   $

\hspace{2em} $\wedge\ s.\textrm{CAFE} = f.\textrm{CAFE}$

\hspace{2em} $\wedge\ s.\textrm{COFFEE} = \textrm{\textquotesingle{Cold} Brew\textquotesingle}) \}$

	
That might lead us to think that we could answer:

\textbf{Who has not gone to a cafe serving Cold Brew?}

With 

$\{f.\textrm{DRINKER } | \textrm{ FREQUENTS}(f) \wedge \neg \exists(s)(\textrm{SERVES}(s)   $

\hspace{2em} $\wedge\ s.\textrm{CAFE} = f.\textrm{CAFE}$

\hspace{2em} $\wedge\ s.\textrm{COFFEE} = \textrm{\textquotesingle{Cold} Brew\textquotesingle}) \}$


\newpage

\hspace{-2em}However, 

$\{f.\textrm{DRINKER } | \textrm{ FREQUENTS}(f) \wedge \neg \exists(s)(\textrm{SERVES}(s) \wedge\ s.\textrm{CAFE} = f.\textrm{CAFE}\wedge\ s.\textrm{COFFEE} = \textrm{\textquotesingle{Cold} Brew\textquotesingle}) \}$

\hspace{-2em}Actually returns: \textbf{Who has gone to a cafe that does not serve Cold Brew?}\\

\hspace{-2em}For convenience, let's say:
 
hasCB $ = \textrm{SERVES}(s) \wedge\ s.\textrm{CAFE} = f.\textrm{CAFE}\wedge\ s.\textrm{COFFEE} = \textrm{\textquotesingle{Cold} Brew\textquotesingle}$



\hspace{-2em}Let's walk through this  expression.

Consider only the data for Risa


\begin{enumerate}
\item Start with the FREQUENTS table
\item Then look at matches in the SERVES table, where s.CAFE = f.CAFE
\item Evaluate the predicate
\item Is it TRUE?
\begin{itemize}
\item If Yes, then include f.DRINKER in the result set
\end{itemize}
\item When f.CAFE = `B', Risa gets included in the result set.

\hspace{-1em}
\begin{tabular}{|l|l|l|p{2cm}||l|l||l|}  \hline
\textrm{f.DRINKER} & \textrm{f.CAFE} &  \textrm{s.CAFE} &  \textrm{s.COFFEE} &\textrm{hasCB} &  $\neg$\textrm{hasCB} & \textrm{Result Set}\\ \hline
 Risa & B  & B & Drip & F& T & \{Risa\}\\ \hline 
 \end{tabular}

\item When f.CAFE = `A', Risa does NOT get add to the  result set

\hspace{-1em}
\begin{tabular}{|l|l|l|p{2cm}||l|l||l|}  \hline
\textrm{f.DRINKER} & \textrm{f.CAFE} &  \textrm{s.CAFE} &  \textrm{s.COFFEE} &\textrm{hasCB} &  $\neg$\textrm{hasCB} & \textrm{Result Set}\\ \hline
&   & A & Drip &  \multirow{3}{*}T  &  \multirow{3}{*}F &  \multirow{3}{*}{ \{ \}} \\  \cline{3-4}
Risa & A & A & Cold Brew &   & & \\  \cline{3-4}
 &  & A & Espresso &  & & \\ \hline
%\textrm{DRINKER} & \textrm{CAFE} &  \textrm{CAFE} &  \textrm{COFFEE} &  \textrm{Result}\\ \hline
% Risa & B  & B & Drip & T\\ \hline \hline
%Risa & A & A & Cold Brew & \multirow{3}{*}F\\ \cline{1-4}
%Risa&  A & A & Drip & \\ \cline{1-4}
% Risa&  A& A & Espresso &\\ \hline
\end{tabular}

\end{enumerate}

\begin{itemize}
\item To get the final result set, we union together all the results from the final column

\hspace{-1em}
\begin{tabular}{|l|l|l|p{2cm}||l|l||l|}  \hline \hline
\color{white}\textrm{DRINKER} & \color{white}\textrm{CAFE} & \color{white} \textrm{CAFE} &  \color{white}\textrm{COFFEE} &\color{white}\textrm{hasCB} & \color{white} $\neg$\textrm{hasCB} &\color{white} \textrm{Result Set}\\ 

& & & & & & \{ Risa \} \\ \hline
 \end{tabular}

\item The final result set is \{ Risa \} $\cup$ \{ \}  = \{ Risa \}
\item However, Risa shouldn't be in the result set, because she frequents Cafe A, which serves Cold Brew 
\item Issue: We want to look at ALL of the coffees served at ALL of the cafes Risa frequents all at one time
\end{itemize}

\newpage

\textbf{Who has not gone to a cafe serving Cold Brew?}

To answer this question, we need to introduce a second variable:

$\{f_1.\textrm{DRINKER } | \textrm{FREQUENTS}(f_1) \wedge\ \neg \exists(f_2,s)(FREQ(f_2) $

\hspace{2em} $\wedge\ SERVES(s) \wedge f_2.CAFE = s.CAFE $

\hspace{2em} $\wedge\ s.COFFEE =  \textrm{\textquotesingle{Cold Brew}\textquotesingle} \wedge\ f_1.DRINKER = f_2.DRINKER) \}$


Again, for convenience, let's say:
 
hasCB $ =(FREQ(f_2) $

\hspace{2em} $\wedge\ SERVES(s) \wedge f_2.CAFE = s.CAFE $

\hspace{2em} $\wedge\ s.COFFEE =  \textrm{\textquotesingle{Cold Brew}\textquotesingle} \wedge\ f_1.DRINKER = f_2.DRINKER) $



In this case, by having the second variable, we are able to look at all the data for every place Risa frequents as a whole.

\begin{itemize}
\item Here, we have another variable, $f_2$
\item We consider each drinker in turn from the FREQUENTS relation. Basically, we are using this table as our master list of drinkers, and are ignoring the CAFE attribute.

Again, look just at Risa.


\begin{tabular}{|l|}  \hline
\textrm{DRINKER} \\ \hline
Risa \\ \hline
\end{tabular}


\item Now, look at all the combinations of FREQUENTS and SERVES where the CAFE matches and the drinker is $f_1$.DRINKER 

\begin{tabular}{|l|l|l|l|p{2cm}||l|l||l|}  \hline 
\textrm{$f_1$.DRINKER} &\textrm{$f_2$.DRINKER} & \textrm{$f_2$.CAFE} &  \textrm{s.CAFE} &  \textrm{s.COFFEE} &\textrm{hasCB} &  $\neg$\textrm{hasCB} & \textrm{Result Set}\\ \hline
 & Risa & A & A & Cold Brew & \multirow{3}{*}T & \multirow{3}{*}F &  \multirow{3}{*}{\{ \}}\\ \cline{2-5}
Risa &Risa & A  & A & Drip & & & \\ \cline{2-5}
 & Risa & A  & A & Espresso & &  &\\ \cline{2-5}
 & Risa & B  & B & Drip & & & \\ \hline
\end{tabular}

\item If there is any tuple where the Coffee is `Cold Brew', we exclude the drinker


\item Now, in this case, one of the cafes that Risa frequents does serve Cold Brew, so Risa is not added to the result set
\end{itemize}



\end{document}