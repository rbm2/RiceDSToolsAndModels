%Copyright 2019 Christopher M. Jermaine (cmj4@rice.edu) and Risa B. Myers (rbm2@rice.edu)
%
%Licensed under the Apache License, Version 2.0 (the "License");
%you may not use this file except in compliance with the License.
%You may obtain a copy of the License at
%
%    https://www.apache.org/licenses/LICENSE-2.0
%
%Unless required by applicable law or agreed to in writing, software
%distributed under the License is distributed on an "AS IS" BASIS,
%WITHOUT WARRANTIES OR CONDITIONS OF ANY KIND, either express or implied.
%See the License for the specific language governing permissions and
%limitations under the License.%===============================================================%
\documentclass[aspectratio=169]{beamer}
\mode<presentation> 
{
\usetheme[noshadow, minimal,numbers,riceb,nonav]{Rice}

\usefonttheme[onlymath]{serif}
\setbeamercovered{transparent}
}

\useinnertheme{rectangles}
\usepackage{colortbl}

\usepackage[english]{babel}

\usepackage{mathptmx}
\usepackage{helvet}
\usepackage{courier}
\usepackage[T1]{fontenc}
\usepackage{trajan}
\usepackage{ textcomp }
\usepackage{amssymb}

%https://tex.stackexchange.com/questions/20740/symbols-for-outer-joins
\def\ojoin{\setbox0=\hbox{$\bowtie$}%
  \rule[-.02ex]{.25em}{.4pt}\llap{\rule[\ht0]{.25em}{.4pt}}}
\def\leftouterjoin{\mathbin{\ojoin\mkern-5.8mu\bowtie}}
\def\rightouterjoin{\mathbin{\bowtie\mkern-5.8mu\ojoin}}
\def\fullouterjoin{\mathbin{\ojoin\mkern-5.8mu\bowtie\mkern-5.8mu\ojoin}}


\usepackage{listings}

\newenvironment{noindentitemize}
{ \begin{itemize}
 \setlength{\itemsep}{1.5ex}
  \setlength{\parsep}{0pt}   
  \setlength{\parskip}{0pt}
 \addtolength{\leftskip}{-2em}
 }
{ \end{itemize} }

\newenvironment{noindentitemize2}
{ \begin{itemize}
  \setlength{\itemsep}{0ex}
  \setlength{\parskip}{0pt}
  \setlength{\parsep}{0pt}   
  \addtolength{\leftskip}{-2em}  }
{ \end{itemize} }

\lstnewenvironment{SQL}
  {\lstset{
        aboveskip=5pt,
        belowskip=5pt,
        escapechar=!,
        mathescape=true,
        upquote=true,
        language=SQL,
        basicstyle=\linespread{0.94}\ttfamily\footnotesize,
        morekeywords={FOR, EACH, WITH, PARTITION, AND, ALL, TEST, WHETHER, PROBABILITY},
        deletekeywords={VALUE, PRIOR},
        showstringspaces=true}
        \vspace{0pt}
        \noindent\minipage{0.47\textwidth}}
  {\endminipage\vspace{0pt}}


\newcommand{\WHERE}{\texttt{WHERE}} 
\newcommand{\ALL}{\texttt{ALL}} 
\newcommand{\UNION}{\texttt{UNION}} 
\newcommand{\EXCEPT}{\texttt{EXCEPT}} 
\newcommand{\LIKES}{\textrm{LIKES}} 
\newcommand{\FREQUENTS}{\textrm{FREQUENTS}} 
\newcommand{\SERVES}{\textrm{SERVES}} 
\newcommand{\CAFE}{\textrm{CAFE}} 
\newcommand{\COFFEE}{\textrm{COFFEE}} 
\newcommand{\DRINKER}{\textrm{DRINKER}} 
\newcommand{\CB}{\textrm{\textquotesingle{Cold} Brew\textquotesingle}} 
\newcommand{\CBGOOD}{\textrm{CBGOOD}} 
\newcommand{\ALLPEEPS}{\textrm{ALLPEEPS}} 
\newcommand{\ALLCOMBOS}{\textrm{ALLCOMBOS}} 
\newcommand{\NOGOODCOFFEE}{\textrm{NOGOODCOFFEE}} 

\setbeamerfont{block body}{size=\tiny}

%===============================================================%

\title[]
{Tools \& Models for Data Science}

\subtitle{SQL Set Operations and Subqueries}

\author[]{Risa Myers \& Chris Jermaine}
\institute
{
  Rice University
}

\date[]{}


\begin{document}

\begin{frame}
 \titlepage
\end{frame}


%***********************************************************
\begin{frame}{Set Operations in SQL}

\begin{itemize}
\item Results are unordered multisets/bag
\item It could be useful to perform operations on these 
\begin{itemize}
\item Union
\item Intersection
\item Difference
\end{itemize}
\item Different RDBMs provide different levels of support

\end{itemize}
\end{frame}


%***********************************************************
\begin{frame}{\UNION\ and \UNION\ \ALL}

\begin{itemize}
\item \UNION - eliminates duplicates
\item \UNION\ \ALL - does NOT eliminate duplicates
\item Uses the column names from the first result set
\item Data types must match
\item Number of attributes must match
\end{itemize}
\end{frame}

%***********************************************************
\begin{frame}[fragile]{\UNION\ and \UNION\ \ALL\ Example}

STUDENT(\underline{NETID}, FIRSTNAME, LASTNAME)\\
FACULTY(\underline{NETID}, FIRSTNAME, LASTNAME)

\begin{SQL}
SELECT lastName, firstName, 'student' 
FROM Student
UNION 
SELECT lastName, firstName, 'faculty' 
FROM Faculty;
\end{SQL}

\end{frame}


%***********************************************************
\begin{frame}{Intersection and Difference}

\begin{itemize}
\item Intersection - Implemented via \texttt{INNER JOIN}
\item Difference - Implemented via \texttt{EXCEPT} 
\end{itemize}
\end{frame}

%***********************************************************
\begin{frame}[fragile]{SELECT-FROM-{\textcolor{red}{WHERE}}}

\begin{SQL}
SELECT <attribute list>
FROM <tables>
WHERE <conditions>

SELECT * 
FROM FREQUENTS f
WHERE f.drinker = 'Risa'
\end{SQL}

\begin{tabular}{|l|l| }  \hline
\textrm{DRINKER} & \textrm{CAFE} \\ \hline
Risa & Double Trouble \\ \hline
Risa & Java Lava  \\ \hline
\end{tabular}


\end{frame}
%***********************************************************
\begin{frame}{\WHERE\ Clause}

\begin{enumerate}
\item $<$attribute$>$ = <value>
\item $<$attribute$>$ BETWEEN [value1] AND [value2]
\item $<$attribute$>$ IN ([value1], [value2], $\ldots$)
\item $<$attribute$>$ LIKE \textquotesingle{SST\%}\textquotesingle
\item $<$attribute$>$ LIKE \textquotesingle{SST\_}\textquotesingle
\item $<$attribute$>$ IS NULL and [attribute] IS NOT NULL
\item Logical combinations with AND and OR
\item Mathematical functions <>, !=, >, <, $\ldots$
\item Subqueries $\ldots$

\end{enumerate}
\end{frame}




%***********************************************************
\begin{frame}{Subqueries}

\begin{itemize}
\item We can have a subquery in the \texttt{WHERE} clause
\item It's linked with keywords
	\begin{itemize}
	\item \texttt{EXISTS} /  \texttt{NOT EXISTS} 
	\begin{itemize}
	\item If the subquery returns at least one tuple, the EXISTS clause evaluates to TRUE
	\end{itemize}
	\end{itemize}
	
\end{itemize}
\end{frame}
%***********************************************************
\begin{frame}{Subqueries}

\begin{itemize}
\item We can have a subquery in the \texttt{WHERE} clause
\item It's linked with keywords
	\begin{itemize}
	\item {<operand>} \texttt{IN} / <operand> \texttt{NOT IN}
	\end{itemize}
	
\item[?] How does IN work? (How else could the expression be written?)
\end{itemize}
\end{frame}
%***********************************************************
\begin{frame}{Subqueries}

\begin{itemize}
\item We can have a subquery in the \texttt{WHERE} clause
\item It's linked with keywords
	\begin{itemize}
	\item {<operand>} \texttt{IN} / <operand> \texttt{NOT IN}
	\end{itemize}
	
\item How does IN work? (How else could the expression be written?)
\item Logical OR of the operand and each value returned by the subquery
\end{itemize}
\end{frame}



%***********************************************************
\begin{frame}{Subqueries}

\begin{itemize}
\item We can have a subquery in the \texttt{WHERE} clause
\item It's linked with keywords
	\begin{itemize}
	\item {<operand>} <comparison operator> \texttt{ALL}
	\item {<operand>} <comparison operator> \texttt{SOME/ANY}
	\end{itemize}
	
\item[?] What is meant by an operand, in this context?
\end{itemize}
\end{frame}
%***********************************************************
\begin{frame}{Subqueries}

\begin{itemize}
\item We can have a subquery in the \texttt{WHERE} clause
\item It's linked with keywords
	\begin{itemize}
	\item {<operand>} <comparison operator> \texttt{ALL}
	\item {<operand>} <comparison operator> \texttt{SOME/ANY}
	\end{itemize}
	
\item What is meant by an operand, in this context?
\item An operand could be an attribute, a function or even a constant
\end{itemize}
\end{frame}


%***********************************************************
\begin{frame}{Subqueries - How do They Work?}

\begin{itemize}
\item Basically, we iterate over the tuples in the outer query and evaluate the inner query for each outer tuple
\item Some can be evaluated once and the result is used in the outer query
	\begin{itemize}
	\item Ex: a subquery that returns the number of CAFES that are frequented
	\end{itemize}

\item Some require the subquery to be evaluated for every value assignment in the outer query (correlated subquery)
	\begin{itemize}
	\item Ex: a subquery that returns the number of CAFES that each  DRINKER frequents
	\end{itemize}

\end{itemize}
\end{frame}


%***********************************************************
\begin{frame}[fragile]{Subquery Example 1 \texttt{IN}}

LIKES (DRINKER, COFFEE)

\begin{noindentitemize}
\item Who likes \textquotesingle{Cold Brew}\textquotesingle and \textquotesingle{Espresso}\textquotesingle?
\item Both subqueries return the same result
\item Many (all?) subqueries can be written as JOINS, people tend to find it easier to reason about one way or the other
\end{noindentitemize}

\begin{columns}[T]
\begin{column}{0.5\textwidth}
\begin{SQL}
SELECT DISTINCT l.DRINKER
FROM LIKES l
WHERE l.COFFEE = 'Cold Brew' 
	AND l.DRINKER IN (
		SELECT l2.DRINKER 
		FROM LIKES l2 
		WHERE l2.COFFEE = 'Espresso')
\end{SQL}
\end{column}
\begin{column}{0.5\textwidth}
\begin{SQL}
SELECT DISTINCT l1.DRINKER
FROM LIKES l1, LIKES l2
WHERE l1.DRINKER = l2.DRINKER 
	AND l1.COFFEE = 'Cold Brew'     
	AND l2.COFFEE = 'Espresso'
\end{SQL}
\end{column}
\end{columns}

\end{frame}
%***********************************************************
\begin{frame}[fragile]{Subquery Example 2 \texttt{EXISTS}}

LIKES (DRINKER, COFFEE)

\begin{noindentitemize}
\item[?] Who goes to a cafe that serves \textquotesingle{Cold Brew}\textquotesingle?
\end{noindentitemize}

\begin{columns}[T]
\begin{column}{0.5\textwidth}
\begin{SQL}
SELECT DISTINCT f.DRINKER
FROM FREQUENTS f, SERVES s
WHERE f.CAFE = s.CAFE
	AND s.COFFEE = 'Cold Brew'
\end{SQL}
\end{column}
\begin{column}{0.5\textwidth}
\begin{SQL}
SELECT DISTINCT f.DRINKER
FROM FREQUENTS f
WHERE EXISTS (
	-- Your code here
	)
\end{SQL}
\end{column}
\end{columns}

\end{frame}


%***********************************************************
\begin{frame}[fragile]{Subquery Example 2 \texttt{EXISTS}}

LIKES (DRINKER, COFFEE)

\begin{noindentitemize}
\item Who goes to a cafe that serves \textquotesingle{Cold Brew}\textquotesingle?
\end{noindentitemize}

\begin{columns}[T]
\begin{column}{0.5\textwidth}
\begin{SQL}
SELECT DISTINCT f.DRINKER
FROM FREQUENTS f, SERVES s
WHERE f.CAFE = s.CAFE
	AND s.COFFEE = 'Cold Brew'
\end{SQL}
\end{column}
\begin{column}{0.5\textwidth}
\begin{SQL}
SELECT DISTINCT f.DRINKER
FROM FREQUENTS f
WHERE EXISTS (
	SELECT s.CAFE
	FROM SERVES s
	WHERE s.COFFEE = 'Cold Brew' 
	   AND f.CAFE = s.CAFE)
\end{SQL}
\end{column}
\end{columns}

\end{frame}



%***********************************************************
\begin{frame}{Subquery Example 3}

LIKES (DRINKER, COFFEE)

FREQUENTS (DRINKER, CAFE)

SERVES (CAFE, COFFEE)


\begin{noindentitemize}
\item[?] Who likes all of the coffees that Risa likes?
\end{noindentitemize}
\end{frame}

%***********************************************************
\begin{frame}[fragile]{Subquery Example 3: Relational Calculus}

LIKES (DRINKER, COFFEE)

FREQUENTS (DRINKER, CAFE)

SERVES (CAFE, COFFEE)


\begin{noindentitemize}
\item[?] Who likes all of the coffees that Risa likes?
\item There doesn't exist a coffee Risa likes that is not also liked by these drinkers
\item Every coffee Risa likes is liked by these drinkers BUT they might like other coffees as well
\end{noindentitemize}



\begin{noindentitemize}
\item Same as:
	%\begin{noindentitemize}
	\item[] $\{l.\DRINKER | \LIKES(l) \wedge \neg \exists(l_2)(\LIKES(l_2) \wedge\ 
	   l_2.\DRINKER = \textrm{\textquotesingle{Risa}\textquotesingle}$\\

           \hspace{2em}$\wedge\ \neg \exists(l_3)(\LIKES(l_3) \wedge\
		l_3.\DRINKER = l.\DRINKER $\\
	\hspace{2em}$\wedge l_3.\COFFEE = l_2.\COFFEE))\}$
%	\end{noindentitemize}
\end{noindentitemize}
\end{frame}


%***********************************************************
\begin{frame}[fragile]{Subquery Example 3 Component}

LIKES (DRINKER, COFFEE)

FREQUENTS (DRINKER, CAFE)

SERVES (CAFE, COFFEE)


\begin{noindentitemize}
\item[?] Coffees that Risa likes
\end{noindentitemize}
\end{frame}

%***********************************************************
\begin{frame}[fragile]{Subquery Example 3 Component}

LIKES (DRINKER, COFFEE)

FREQUENTS (DRINKER, CAFE)

SERVES (CAFE, COFFEE)


\begin{noindentitemize}
\item Coffees that Risa likes
\end{noindentitemize}

\begin{SQL}
SELECT l2.COFFEE
FROM LIKES l2
WHERE l2.DRINKER = 'Risa' 
\end{SQL}
\end{frame}



%***********************************************************
\begin{frame}[fragile]{Subquery Example 3}

LIKES (DRINKER, COFFEE)

FREQUENTS (DRINKER, CAFE)

SERVES (CAFE, COFFEE)


\begin{noindentitemize}
\item[?] Who likes all of the coffees that Risa likes?
\end{noindentitemize}

\begin{SQL}
SELECT DISTINCT l.DRINKER
FROM LIKES l
WHERE NOT EXISTS ($\textrm{a coffee Risa likes that is not also liked by l.DRINKER}$)
\end{SQL}
\begin{noindentitemize}
	%\begin{noindentitemize}
	\item[] $\{l.\DRINKER | \LIKES(l) \wedge \neg \exists(l_2)(\LIKES(l_2) \wedge\ 
	   l_2.\DRINKER = \textrm{\textquotesingle{Risa}\textquotesingle}$\\

           \hspace{2em}$\wedge\ \neg \exists(l_3)(\LIKES(l_3) \wedge\
		l_3.\DRINKER = l.\DRINKER $\\
	\hspace{2em}$\wedge l_3.\COFFEE = l_2.\COFFEE))\}$
%	\end{noindentitemize}
\end{noindentitemize}


\end{frame}

%***********************************************************
\begin{frame}[fragile]{Subquery Example 3}

LIKES (DRINKER, COFFEE)

FREQUENTS (DRINKER, CAFE)

SERVES (CAFE, COFFEE)


\begin{noindentitemize}
\item[?] Who likes all of the coffees that Risa likes?
\end{noindentitemize}

\begin{SQL}
SELECT DISTINCT l.DRINKER
FROM LIKES l
WHERE NOT EXISTS (
  SELECT l2.COFFEE 
  FROM LIKES l2
  WHERE l2.DRINKER = 'Risa' AND l2.COFFEE NOT IN (
    $\textrm{the set of coffees liked by l.DRINKER}$))
\end{SQL}

\begin{noindentitemize}
\item There doesn't exist a coffee that Risa likes where that coffee is not \ldots
	%\begin{noindentitemize}
	\item[] $\{l.\DRINKER | \LIKES(l) \wedge \neg \exists(l_2)(\LIKES(l_2) \wedge\ 
	   l_2.\DRINKER = \textrm{\textquotesingle{Risa}\textquotesingle}$\\

           \hspace{2em}$\wedge\ \neg \exists(l_3)(\LIKES(l_3) \wedge\
		l_3.\DRINKER = l.\DRINKER $\\
	\hspace{2em}$\wedge l_3.\COFFEE = l_2.\COFFEE))\}$
%	\end{noindentitemize}
\end{noindentitemize}



\end{frame}


%***********************************************************
\begin{frame}[fragile]{Subquery Example 3 Final}

LIKES (DRINKER, COFFEE)

FREQUENTS (DRINKER, CAFE)

SERVES (CAFE, COFFEE)


\begin{noindentitemize}
\item[?] Who likes all of the coffees that Risa likes?
\item Still need: A coffee Risa likes that is not also liked by l.DRINKER
\end{noindentitemize}

\begin{SQL}
SELECT DISTINCT l.DRINKER
FROM LIKES l
WHERE NOT EXISTS (
  SELECT l2.COFFEE 
  FROM LIKES l2
  WHERE l2.DRINKER = 'Risa' 
    AND that coffee is also not liked by l.DRINKER)
\end{SQL}
\end{frame}

%***********************************************************
\begin{frame}[fragile]{Subquery Example 3 Final}

LIKES (DRINKER, COFFEE)

FREQUENTS (DRINKER, CAFE)

SERVES (CAFE, COFFEE)


\begin{noindentitemize}
\item[?] Who likes all of the coffees that Risa likes?
\item Still need: A coffee Risa likes that is not also liked by l.DRINKER
\end{noindentitemize}

\begin{SQL}
SELECT DISTINCT l.DRINKER
FROM LIKES l
WHERE NOT EXISTS (
  SELECT l2.COFFEE 
  FROM LIKES l2
  WHERE l2.DRINKER = 'Risa' 
    AND l2.COFFEE NOT IN (coffees liked by l.DRINKER)
\end{SQL}
\end{frame}


%***********************************************************
\begin{frame}[fragile]{Subquery Example 3 Final}

LIKES (DRINKER, COFFEE)

FREQUENTS (DRINKER, CAFE)

SERVES (CAFE, COFFEE)


\begin{noindentitemize}
\item[?] Who likes all of the coffees that Risa likes?
\item Still need: A coffee Risa likes that is not also liked by l.DRINKER
\end{noindentitemize}

\begin{SQL}
SELECT DISTINCT l.DRINKER
FROM LIKES l
WHERE NOT EXISTS (
  SELECT l2.COFFEE 
  FROM LIKES l2
  WHERE l2.DRINKER = 'Risa' 
    AND l2.COFFEE NOT IN (
      SELECT l3.COFFEE
      FROM LIKES l3
      WHERE l3.DRINKER = l.DRINKER))
\end{SQL}
\end{frame}

%***********************************************************
\begin{frame}[fragile]{Subquery Example 3 vs. Relational Calculus}

\begin{SQL}
SELECT DISTINCT l.DRINKER
FROM LIKES l
WHERE NOT EXISTS (
  SELECT l2.COFFEE 
  FROM LIKES l2
  WHERE l2.DRINKER = 'Risa' AND l2.COFFEE NOT IN (
    SELECT l3.COFFEE
    FROM LIKES l3
    WHERE l3.DRINKER = l.DRINKER))
\end{SQL}

\begin{noindentitemize}
\item l.DRINKER = l3.DRINKER
\item l2.DRINKER = `Risa'
\item Same as:
	%\begin{noindentitemize}
	\item[] $\{l.\DRINKER | \LIKES(l) \wedge \neg \exists(l_2)(\LIKES(l_2) \wedge\ 
	   l_2.\DRINKER = \textrm{\textquotesingle{Risa}\textquotesingle}$\\

           \hspace{2em}$\wedge\ \neg \exists(l_3)(\LIKES(l_3) \wedge\
		l_3.\DRINKER = l.\DRINKER $\\
	\hspace{2em}$\wedge l_3.\COFFEE = l_2.\COFFEE))\}$
%	\end{noindentitemize}
\end{noindentitemize}
\end{frame}

%***********************************************************
\begin{frame}{\texttt{SOME}/\texttt{ANY} predicate}


\begin{noindentitemize}
\item \texttt{SOME}/\texttt{ANY} is used like ``\texttt{expression boolOp \{SOME, ANY \}(subquery)}''
\item \texttt{SOME}/\texttt{ANY} returns TRUE if there is at least 1 item in the subquery can make the boolOp evaluate to true
\end{noindentitemize}
\end{frame}

% SOME vs EXISTS - WHERE EXISTS vs comparison with SOME / ANY

%***********************************************************
\begin{frame}{\texttt{SOME}/\texttt{ANY} Example}

Given the relation:\\
RATES (DRINKER, COFFEE, SCORE)\\

\begin{noindentitemize}
\item Ratings go from low to high, with increasing values indicating higher levels of liking the coffee.
\item[?] Of the coffees Risa has rated, list the coffees that are not Risa's favorite. 
\item[?] What does it mean, in terms of RATES, when we say favorite?
\end{noindentitemize}

\end{frame}

%***********************************************************
\begin{frame}[fragile]{\texttt{SOME}/\texttt{ANY} Example}

Given the relation:\\
RATES (DRINKER, COFFEE, SCORE)\\

\begin{itemize}
\item Of the coffees Risa has rated, list the coffees that are not Risa's favorite. 
\end{itemize}


\begin{SQL}
SELECT r.COFFEE
FROM RATES r
WHERE r.DRINKER = 'Risa' AND r.SCORE < SOME (
  SELECT r2.SCORE 
  FROM RATES r2
  WHERE r2.DRINKER = 'Risa' )
\end{SQL}

\end{frame}

%***********************************************************
\begin{frame}[fragile]{\texttt{SOME}/\texttt{ANY} Example Unpacked}

Given the relation:\\
RATES (DRINKER, COFFEE, SCORE)\\

\begin{itemize}
\item Of the coffees Risa has rated, list the coffees that are not Risa's favorite. 
\end{itemize}


\begin{SQL}
SELECT r.COFFEE
FROM RATES r
WHERE r.DRINKER = 'Risa' AND r.SCORE < SOME (
  SELECT r2.SCORE 
  FROM RATES r2
  WHERE r2.DRINKER = 'Risa' )
\end{SQL}

\begin{itemize}
\item The subquery returns the multiset of all the scores that Risa has given to coffees
\item The r.SCORE < SOME clause evaluates to TRUE if the multiset is not empty
\end{itemize}


\end{frame}


%***********************************************************
\begin{frame}[fragile]{\ALL\ predicate}


\begin{noindentitemize}
\item \texttt{ALL} is used like ``\texttt{expression boolOp ALL (subquery)}''
\item Similar to \texttt{SOME}
\item BoolOp must evaluate to true for \textbf{everything} in the subquery
\end{noindentitemize}

\end{frame}

%***********************************************************
\begin{frame}[fragile]{\ALL\ Example}

RATES (DRINKER, COFFEE, SCORE)

\begin{SQL}
SELECT DISTINCT r.DRINKER
FROM RATES r
WHERE r.SCORE < ALL (
 SELECT r2.SCORE
  FROM RATES r2
  WHERE r2.DRINKER = 'Risa')
\end{SQL}

\begin{itemize}
\item[?] What does this query return? %Drinkers who rated a coffee lower than all of Risa's scores
\end{itemize}
\end{frame}

%***********************************************************
\begin{frame}[fragile]{\ALL\ Example}

RATES (DRINKER, COFFEE, SCORE)

\begin{SQL}
SELECT DISTINCT r.DRINKER
FROM RATES r
WHERE r.SCORE < ALL (
 SELECT r2.SCORE
  FROM RATES r2
  WHERE r2.DRINKER = 'Risa')
\end{SQL}

\begin{itemize}
\item What does this query return? 
\item Drinkers who rated a coffee lower than all of Risa's scores
\end{itemize}
\end{frame}


%***********************************************************
\begin{frame}[fragile]{Subqueries in \texttt{FROM} Clause}

FREQUENTS (DRINKER, CAFE)

\begin{itemize}
\item Can have a subquery in FROM clause
\item Treated as a temporary table
\item MUST be assigned an alias
\item[?] Who goes to a cafe that serves 'Cold Brew'?
\end{itemize}

\begin{columns}[T]
\begin{column}{0.5\textwidth}
Old way\\
\begin{SQL}
SELECT DISTINCT f.DRINKER
FROM FREQUENTS f, SERVES s
WHERE f.CAFE = s.CAFE 
	AND s.COFFEE = 'Cold Brew'
\end{SQL}
\end{column}
\begin{column}{0.5\textwidth}
New way\\
\begin{SQL}
SELECT DISTINCT f.DRINKER
FROM FREQUENTS f, 
   (SELECT s.CAFE FROM SERVES s 			
    WHERE s.COFFEE = 'Cold Brew') s2
WHERE f.CAFE = s2.CAFE
\end{SQL}
\end{column}
\end{columns}
\end{frame}


%***********************************************************
\begin{frame}[fragile]{Subquery in \texttt{FROM} Clause}

FREQUENTS (DRINKER, CAFE)

\begin{noindentitemize}
\item Note: The code is a lot cleaner with a view!
\end{noindentitemize}

\begin{SQL}
CREATE VIEW CB_COFFEE AS
SELECT s.CAFE FROM SERVES s 			
    WHERE s.COFFEE = 'Cold Brew'
\end{SQL}

\begin{SQL}
SELECT DISTINCT f.DRINKER
FROM FREQUENTS f, CB_COFFEE c
WHERE f.CAFE = c.CAFE
\end{SQL}
\end{frame}

%***********************************************************
\begin{frame}[fragile]{Views}

\begin{noindentitemize}
\item ``Common'' (non-materialized) views are just macros
\item Query definition
\item Unexecuted query 
\item Virtual table
\item Can be used in place of a table
\item Convenient way to simplify a query
\item Is executed when called (results are NOT stored)

\item[?] List the coffees that are not Risa's favorite.
\end{noindentitemize}

\end{frame}

%***********************************************************
\begin{frame}[fragile]{Views}

\begin{noindentitemize}
\item ``Common'' (non-materialized) views are just macros
\item List the coffees that are not Risa's favorite.
\end{noindentitemize}

\begin{SQL}
CREATE VIEW RISA_COFFEES AS
SELECT * 
FROM RATES r
WHERE r.DRINKER = 'Risa'
\end{SQL}

\begin{SQL}
SELECT r.COFFEE
FROM RISA_COFFEES r
WHERE r.SCORE < SOME (
  SELECT r2.SCORE
  FROM RISA_COFFEES r2)
\end{SQL}
\end{frame}

%***********************************************************
\begin{frame}[fragile]{Materialized Views}

\begin{noindentitemize}
\item[?] What's different?
\end{noindentitemize}
\end{frame}
%***********************************************************
\begin{frame}[fragile]{Materialized Views}

\begin{noindentitemize}
\item What's different?
\begin{noindentitemize2}
\item Query is run when the view is created
\item Results are stored until the view is \texttt{REFRESHED}
\end{noindentitemize2}
\end{noindentitemize}

\end{frame}


%***********************************************************
\begin{frame}{Some Notes}

\begin{itemize}
	\item Declarative SQL code tends to be very short
	\item Good: because effort \& bugs $\propto$ code length
	\item Bad: because it can be difficult to understand!
\end{itemize}
\end{frame}

%***********************************************************
\begin{frame}{Some Notes on Style}

\begin{itemize}
\item Hence, style is important.  Some suggestions
	\begin{itemize}
	\item Always alias tuple variables and relations
	\item Always indent carefully
	\item Only one major keyword per line (\texttt{SELECT}, \texttt{FROM}, etc.)
	\item Pick a capitalization scheme and religiously stick to it
	\item Make frequent use of views...
	\end{itemize}
\end{itemize}
\end{frame}

%***********************************************************
\begin{frame}{Defensive SQL}

\begin{itemize}
\item Explicitly list all attributes
\item Specify the relation attribute for each attribute
\item Left  outer join / Right outer join - CHOOSE 1
\end{itemize}
\end{frame}

%***********************************************************
\begin{frame}{True/False Questions}

\begin{enumerate}
\item Every SQL query must contain a WHERE clause %F
\item  By default, a VIEW stores the data retrieved from the query %F
\item Subqueries  isolate parts of queries % T
\item Using VIEWs makes it harder to figure out what's going on in a query %F
\item Subqueries can only return a single value % F in ex 1 we returned a list of drinkers
\item Subqueries must reference an attribute from the outer query %F they may
\item Subqueries can only appear in the WHERE clause % F, can be in FROM clause as well
\end{enumerate}

\end{frame}


%***********************************************************

\begin{frame}{Wrap up}
\begin{itemize}
	\item[?] How can we use what we learned today?
	\vspace{2em}
	\item[?] What do we know now that we didn't know before?
\end{itemize}

\end{frame}

\end{document}
