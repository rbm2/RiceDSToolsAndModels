%Copyright 2019 Christopher M. Jermaine (cmj4@rice.edu) and Risa B. Myers (rbm2@rice.edu)
%
%Licensed under the Apache License, Version 2.0 (the "License");
%you may not use this file except in compliance with the License.
%You may obtain a copy of the License at
%
%    https://www.apache.org/licenses/LICENSE-2.0
%
%Unless required by applicable law or agreed to in writing, software
%distributed under the License is distributed on an "AS IS" BASIS,
%WITHOUT WARRANTIES OR CONDITIONS OF ANY KIND, either express or implied.
%See the License for the specific language governing permissions and
%limitations under the License.
%===============================================================
\documentclass[aspectratio=169]{beamer}
\mode<presentation> 
{
\usetheme[noshadow, minimal,numbers,riceb,nonav]{Rice}
\usefonttheme[onlymath]{serif}
\setbeamercovered{transparent}
}
\useinnertheme{rectangles}

\usepackage[english]{babel}

\usepackage{mathptmx}
\usepackage{helvet}
\usepackage{courier}
\usepackage[T1]{fontenc}
\usepackage{trajan}
\usepackage{ textcomp }
\usepackage{listings}

\newenvironment{noindentitemize}
{ \begin{itemize}
 \setlength{\itemsep}{1.5ex}
  \setlength{\parsep}{0pt}   
  \setlength{\parskip}{0pt}
 \addtolength{\leftskip}{-2em}
 }
{ \end{itemize} }

\newenvironment{noindentitemize2}
{ \begin{itemize}
  \setlength{\itemsep}{0ex}
  \setlength{\parskip}{0pt}
  \setlength{\parsep}{0pt}   
  \addtolength{\leftskip}{-2em}  }
{ \end{itemize} }



\lstnewenvironment{SQL}
  {\lstset{
        aboveskip=5pt,
        belowskip=5pt,
        escapechar=!,
        mathescape=true,
        upquote=true,
        language=SQL,
        basicstyle=\linespread{0.94}\ttfamily\footnotesize,
        morekeywords={PRINT, CURSOR, OPEN, FETCH, CLOSE, DECLARE, BEGIN, END, PROCEDURE, FOR, EACH, WITH, PARTITION, 	TEST, WHETHER, PROBABILITY, OUT,LOOP,IF,CONTINUE, HANDLER,CALL, FUNCTION, RETURNS, LANGUAGE,BODY,RETURN, REPLACE,plpgsql,
        RAISE, NOTICE,
        REPLACE, ROW, BEFORE, EXIT, TEXT, REFCURSOR, QUOTE_LITERAL, DELIMITER,CONCAT,FOUND,LEAVE },
        deletekeywords={VALUE, PRIOR},
        showstringspaces=true}
        \vspace{0pt}%
        \noindent\minipage{0.65\textwidth}}
  {\endminipage\vspace{0pt}}
  
  
\lstnewenvironment{SQLtiny}
  {\lstset{
        aboveskip=5pt,
        belowskip=5pt,
        escapechar=!,
        mathescape=true,
        upquote=true,
        language=SQL,
        basicstyle=\linespread{0.94}\ttfamily\tiny,
        morekeywords={PRINT, CURSOR, OPEN, FETCH, CLOSE, DECLARE, BEGIN, END, PROCEDURE, FOR, EACH, WITH, PARTITION, 	TEST, WHETHER, PROBABILITY, OUT,LOOP,IF,CONTINUE, HANDLER,CALL, FUNCTION, RETURNS, LANGUAGE,BODY,RETURN, REPLACE,plpgsql,
        RAISE, NOTICE,
        REPLACE, ROW, BEFORE, EXIT, TEXT, REFCURSOR, QUOTE_LITERAL, DELIMITER,CONCAT,FOUND,LEAVE },
       deletekeywords={VALUE, PRIOR},
        showstringspaces=true}
        \vspace{0pt}
        \noindent\minipage{0.47\textwidth}}
  {\endminipage\vspace{0pt}}


\newcommand{\LIKES}{\textrm{LIKES}} 
\newcommand{\FREQUENTS}{\textrm{FREQUENTS}} 
\newcommand{\SERVES}{\textrm{SERVES}} 
\newcommand{\CAFE}{\textrm{CAFE}} 
\newcommand{\COFFEE}{\textrm{COFFEE}} 
\newcommand{\DRINKER}{\textrm{DRINKER}} 
\newcommand{\ALLPEEPS}{\textrm{ALLPEEPS}} 
\newcommand{\ALLCOMBOS}{\textrm{ALLCOMBOS}} 
\newcommand{\NOGOODCOFFEE}{\textrm{NOGOODCOFFEE}} 


\setbeamerfont{block body}{size=\tiny}

%===============================================================%

\title[]
{ Tools \& Models for Data Science}

\subtitle{Imperative SQL 2}

\author[]{Chris Jermaine \& Risa Myers}
\institute
{
  Rice University 
}

\date[]{}

\subject{Beamer}


\begin{document}
\begin{frame}
 \titlepage
\end{frame}
%***********************************************************

\begin{frame}[fragile]{More Interesting Cursor Example}

\begin{noindentitemize}
\item[?] Write a stored procedure giving the name and height of the tallest peak in a region.
\end{noindentitemize}

\begin{SQL}
CREATE OR REPLACE FUNCTION  
   getTallestPeak(whichRegion TEXT, 
      OUT finalBestName TEXT, finalBestHeight INTEGER) AS
$\textbf{\$\$}$
BEGIN
...
END;
$\textbf{\$\$}$
LANGUAGE plpgsql ;
\end{SQL}
\end{frame}

%***********************************************************
\begin{frame}[fragile]{More Interesting Cursor Example}

\begin{noindentitemize}
\item Body 1: Declare control variables as well as the cursor
\end{noindentitemize}

\begin{SQL}
CREATE OR REPLACE FUNCTION  
   getTallestPeak(whichRegion TEXT, 
      OUT finalBestName TEXT, finalBestHeight INTEGER) AS
$\textbf{\$\$}$
DECLARE 
   peakName TEXT;
   bestName TEXT;
   peakHeight INTEGER DEFAULT -1;
   bestHeight INTEGER DEFAULT -1;
   peakCursor CURSOR FOR SELECT name, elev 
	FROM Peak WHERE Region = whichRegion;
BEGIN
END;
$\textbf{\$\$}$
LANGUAGE plpgsql;\end{SQL}
\end{frame}

%***********************************************************
\begin{frame}[fragile]{More Interesting Cursor Example}

\begin{noindentitemize}
\item Body 2: Open cursor and loop to find the tallest peak
\end{noindentitemize}

\begin{SQL}   
BEGIN
OPEN peakCursor;
LOOP
	FETCH peakCursor INTO peakname, peakHeight;	
	EXIT WHEN NOT FOUND;
	IF peakHeight > bestHeight THEN
		bestHeight = peakHeight;
		bestName = peakName;
	END IF;
END LOOP;
CLOSE peakCursor;
\end{SQL}
\end{frame}

%***********************************************************

\begin{frame}[fragile]{More Interesting Cursor Example}

\begin{noindentitemize}
\item Body 3: return the result
\end{noindentitemize}

\begin{SQL}
finalBestName = bestName;
finalBestHeight = bestHeight;
END;
$\textbf{\$\$}$
LANGUAGE plpgsql;
\end{SQL}

\begin{noindentitemize}
\item To call:
\end{noindentitemize}

\begin{SQL}
SELECT * FROM getTallestPeak('Corocoran to Whitney');
\end{SQL}

or 

\begin{SQL}
SELECT getTallestPeak('Corocoran to Whitney');
\end{SQL}
\end{frame}


%***********************************************************

\begin{frame}{Important: Don't EVER Write Such Code!}

\begin{itemize}
\item I wrote code that looped to find the tallest
	\begin{itemize}
	\item Terrible idea!
	\item LIMIT k would be shorter, easier, faster
	\item Rule: use AS MUCH declarative code as possible
	\item Only use loops, etc. when you MUST
	\item Sometimes 3+ orders of magnitude speed difference
	\end{itemize}	
\end{itemize}
\end{frame}


%***********************************************************

\begin{frame}[fragile]{Implicit Cursor Version}

\begin{SQL}
CREATE OR REPLACE FUNCTION  
   getTallestPeakImplicit(whichRegion TEXT, OUT finalBestName TEXT, 
      OUT finalBestHeight INTEGER)  AS
$\textbf{\$\$}$
DECLARE 
   bestName TEXT;
   bestHeight INTEGER DEFAULT -1;
   row_data RECORD;
BEGIN
  FOR row_data IN SELECT name, elev FROM Peak WHERE Region = whichRegion
  LOOP
      IF row_data.elev > bestHeight THEN
        bestHeight = row_data.elev;
        bestName = row_data.name;
      END IF;
  END LOOP;
  finalBestName = bestName;
  finalBestHeight = bestHeight;
END;
$\textbf{\$\$}$
LANGUAGE plpgsql;
\end{SQL}
\end{frame}

%***********************************************************

\begin{frame}{Implicit or Explicit?}
\begin{noindentitemize}
\item Implicit cursors are unique(ish?) to Postgres
	\begin{noindentitemize2}
	\item Less portable
	\end{noindentitemize2}
\item But they are easier to use
	\item Limitations?
	\begin{noindentitemize2}
	 \item One direction only (?)
	\end{noindentitemize2}
\end{noindentitemize}

\end{frame}

%***********************************************************
\begin{frame}[fragile]{Returning Relations}

\begin{noindentitemize}
	\item \texttt{RETURNS TABLE}(<list of names, type pairs>)
\end{noindentitemize}

\begin{SQL}
CREATE OR REPLACE FUNCTION  
   tableValuedFunctionPLSQL(x INTEGER)
   RETURNS TABLE (f1 INTEGER, f2 TEXT ) AS
$\textbf{\$\$}$
BEGIN
	RETURN QUERY SELECT x, CAST(x AS TEXT) || ' is text';
END;
$\textbf{\$\$}$
LANGUAGE plpgsql;

SELECT * FROM tableValuedFunctionPLSQL(5)
\end{SQL}

\end{frame}

%***********************************************************
\begin{frame}[fragile]{Returning Relations using SQL language}

\begin{noindentitemize}
	\item \texttt{RETURNS TABLE}(<list of names, type pairs>)
\end{noindentitemize}

\begin{SQL}
CREATE OR REPLACE FUNCTION  
   tableValuedFunction(x INTEGER)
   RETURNS TABLE (f1 INTEGER, f2 TEXT )
AS
$\textbf{\$\$}$
  SELECT x, CAST(x AS TEXT) || ' is text';
$\textbf{\$\$}$
LANGUAGE sql;


SELECT * FROM tableValuedFunction(5)
\end{SQL}

\end{frame}


%***********************************************************
\begin{frame}{Triggers}
\begin{noindentitemize}
	\item Stored procedures that fire in response to some event
	\item Standard options
\begin{noindentitemize2}
	\item Events: UPDATE, INSERT, DELETE
	\item Timing: BEFORE/AFTER: run only once triggering action succeeds
\end{noindentitemize2}
\item Some RDBMSs
\begin{noindentitemize2}
\item Triggers: Typically not run when TRUNCATE is called on a table
\end{noindentitemize2}
\end{noindentitemize}

\end{frame}

%***********************************************************
\begin{frame}{Trigger Special Variables}

\begin{noindentitemize}
	\item old: table containing old versions of records
	\item new: table containing new versions
\end{noindentitemize}

\footnotesize{
\begin{tabular}{|l|l|l|c|l|l|l|c|l|l|l|c|}
    \cline{1-3}  \cline{5-7} \cline{9-11}
    & \multicolumn{2}{c|}{INSERT} & & & \multicolumn{2}{|c|}{UPDATE} & & & \multicolumn{2}{|c|}{DELETE} \\ \cline{2-3}  \cline{6-7} \cline{10-11}
    & Before& After & & & Before& After &  & & Before& After  \\  \cline{1-3} \cline{5-7} \cline{9-11}
    old. & N/A & N/A & & old. & old & old & & old. & old & old\\ \cline{2-3} \cline{5-7} \cline{9-11}
    new. & new & new & & new. & new & new  & & new. & N/A & N/A \\ \cline{2-3} \cline{5-7} \cline{9-11}
    \cline{1-3}  \cline{5-7} \cline{9-11}
\end{tabular}
}
\end{frame}


%***********************************************************
\begin{frame}[fragile]{Trigger Example}

\begin{noindentitemize}
	\item Write a trigger that catches updates to peak table, prints an error message and does not process
	\item Write the function first
	\item Then create a trigger that calls the function
\end{noindentitemize}


\end{frame}

%***********************************************************
\begin{frame}[fragile]{Trigger: Function Framework}

\begin{noindentitemize}
	\item NOTE: Return type of \texttt{TRIGGER}
	\item NOTE: No arguments
\end{noindentitemize}


\begin{SQL}
CREATE OR REPLACE FUNCTION  
   ignorePeakUpdate() RETURNS TRIGGER AS
$\textbf{\$\$}$
DECLARE 
	...
BEGIN
	...
END;
$\textbf{\$\$}$
LANGUAGE plpgsql;
\end{SQL}
\end{frame}

% NOTE RETURN TYPE

%***********************************************************
\begin{frame}[fragile]{Trigger: Function Body}


\begin{SQL}
BEGIN
   RAISE NOTICE 'You changed the height of \% ', old.name ;
   RAISE NOTICE 'from \% to \%.', old.elev, new.elev ;
   RAISE NOTICE 'I am ignoring it.';

   RETURN OLD;
END;
\end{SQL}

\begin{noindentitemize}
	\item NOTE: Multiple lines used here only for formatting reasons. Can all be done on one line
\end{noindentitemize}


\end{frame}
% multiple lines is only for formatting reasons. Can all be done on one line


%***********************************************************
\begin{frame}[fragile]{Trigger Example}

\begin{noindentitemize}
	\item Write a trigger that catches updates to peak table, prints error message and does not process
\end{noindentitemize}

\begin{SQL}
CREATE TRIGGER checkHeight 
BEFORE UPDATE ON peak 
FOR EACH ROW
EXECUTE PROCEDURE ignorePeakUpdate()
\end{SQL}
\end{frame}


%***********************************************************
\begin{frame}[fragile]{Temporary Tables}

\begin{SQL}
CREATE TEMPORARY TABLE myMap (
  myKey INTEGER,
  myValue VARCHAR(200),
  PRIMARY KEY(myKey)
);
\end{SQL}

\begin{itemize}
	\item When to use 
	\begin{itemize}
		\item Passing lots of data
		\item Short life span (current session only)
		\item Debugging
	\end{itemize}
\end{itemize}
\end{frame}

%***********************************************************

\begin{frame}{Wrap up}
\begin{itemize}
	\item[?] How can we use what we learned today?
	\vspace{2em}
	\item[?] What do we know now that we didn't know before?
\end{itemize}
\end{frame}



\end{document}
